% ====================================
% Template NÃO oficial do formulário F-01.1 para cadastro de projeto de pesquisa e inovação do IF Sudeste MG em LaTeX.
% Versão 1.0.0
% Desenvolvido e mantido por:
%   Filipe Fernandes, PhD
%   https://www.linkedin.com/in/filipefernandesphd
%   filipe.fernandes@ifsudestemg.edu.br
% ====================================
\documentclass{article}
\usepackage{template}

% PREENCHA OS DADOS DE IDENTIFICAÇÃO DO PROJETO

% Título do Projeto:
\newcommand{\tituloprojeto}{Título do projeto}

% Área de avaliação WebQualis: (Indicação correta de apenas uma área correspondente ao estrato de avaliação para pontuação no Lattes)
% https://sucupira.capes.gov.br/sucupira/public/consultas/coleta/veiculoPublicacaoQualis/listaConsultaGeralPeriodicos.jsf 
\newcommand{\areaavaliacao}{Área de avaliação}

% Área de Conhecimento CNPq: (Inserir somente o nome da área e não o código numérico)    
% http://www.cnpq.br/documents/10157/186158/TabeladeAreasdoConhecimento.pdf 
\newcommand{\areaconhecimento}{Área de conhecimento}

% Subárea de Conhecimento CNPq: 
% http://www.cnpq.br/documents/10157/186158/TabeladeAreasdoConhecimento.pdf 
\newcommand{\subareaconhecimento}{Subárea de conhecimento}

% Referência do Edital:
\newcommand{\edital}{Edital}

% Campus
\newcommand{\campus}{Campus}

% Tipo(s) de bolsa(s) solicitada(s):
\newcommand{\tiposbolsas}{Tipos de bolsas}

% Quantidade de bolsas solicitadas:
\newcommand{\qtdbolsas}{Quantidade de bolsas}

% Haverá indicação de coorientador?
\newcommand{\coorientador}{( ) Sim ( ) Não}

% Haverá indicação de estudante voluntário?
\newcommand{\voluntario}{( ) Sim ( ) Não}

% Departamento/Núcleo/Curso:
\newcommand{\depnucleocurso}{Departamento/Núcleo/Curso}

% Entidade externa integrante do projeto:
\newcommand{\entidadeexterna}{Entidade externa}

% O projeto será executado por estudantes de diferentes níveis de ensino?
\newcommand{\niveisestudantes}{( ) Sim ( ) Não}

% O projeto foi submetido   ao comitê de ética?
\newcommand{\comiteetica}{( ) Sim ( ) Não ( ) Não se aplica}

% Projeto aprovado em editais anteriores e que caracteriza continuidade?
\newcommand{\editaisanteriores}{( ) Sim ( ) Não}

% Carga horária semanal e semestral prevista para cada participante: 
% Informe somente números
\newcommand{\chsemanal}{Horas}
\newcommand{\chsemestral}{Horas}

\begin{document}

% Página de Identificação do Projeto
\input{idprojeto}

\section{RESUMO:} \label{sec:resumo}
% De forma breve, apresente a justificativa de seu projeto, objetivos e os métodos. 

\section{INTRODUÇÃO:} \label{sec:introdução}
% Apresenta a importância da realização da pesquisa?
% Utiliza citações atualizadas da literatura e argumentos próprios coerentes com a proposta de estudo?
% Apresenta informações suficientes para o entendimento da situação-problema da pesquisa?
% Apresenta uma linguagem clara, objetiva e condizente com o tema de estudo?

\section{RESULTADOS DAS BUSCAS EM BANCOS DE PROPRIEDADE INTELECTUAL:} \label{sec:resultados_patentes}
% Quadro do item 6.6 do Edital.
% Apresente os resultados de buscas em bancos de propriedade intelectual, citando a metodologia de busca - exemplos: palavras-chave, operadores lógicos, classificação, etc.
% Destaque os resultados relevantes que mais se aproximam do objeto proposto no projeto, destacando a diferenciação proposta.
% Consulte o tutorial para consultas às bases de patentes preparado pela PROPPI \url{https://inovare.ifsudestemg.edu.br/administracao/pesquisa/documentos/manual_busca_patentes.pdf}

\section{OBJETIVOS:} \label{sec:objetivos}
% O objetivo geral está formulado de forma clara e bem delimitado?
% É condizente com a questão de pesquisa e coerente com o título do projeto?
% Os objetivos específicos estão definidos claramente e contribuem para o alcance do objetivo geral?

\section{METODOLOGIA:} \label{sec:metodologia}
% Discute de forma clara a natureza da pesquisa (tipo de estudo)?
% Discute o corpus ou a população e define os critérios para definir a amostra e/ou objeto de estudo?
% Apresenta os critérios de inclusão e exclusão para compor a população ou corpus do estudo?
% Em caso de pesquisa documental: apresenta as fontes, detalhando os critérios para a seleção documental?
% Em caso de pesquisa bibliográfica, define os critérios para a seleção dos autores/obras da literatura?
% Explicita o procedimento de levantamento ou coleta de dados?
% Descreve sucintamente as técnicas que serão utilizadas?
% Justifica quais os instrumentos que serão utilizados na coleta de dados?
% As metodologias propostas estão de acordo com os objetivos do trabalho?
% Detalha o processo de análise de dados?
% O processo de análise é coerente com a natureza da pesquisa?

\section{RESULTADOS ESPERADOS/POSSIBILIDADE DE GERAÇÃO DE UM NOVO PRODUTO OU PROCESSO:} \label{sec:resultados_esperados}
% Quadro do item 6.6 do Edital.
% Informa de forma clara os resultados/devolutiva (impacto científico, benefícios aos indivíduos estudados) e/ou produtos (produtos, serviços, processos, métodos e sistemas novos ou significativamente melhorados, publicações, participações em eventos, registro de propriedade intelectual, etc.?

\section{RELAÇÃO COM ENTIDADES EXTERNAS/PARCERIA COM EMPRESAS OU ARRANJOS PRODUTIVOS LOCAIS:} \label{sec:entidades_externas}
% Quadro do item 6.6 do Edital.
% Relevância/inserção/aplicabilidade da tecnologia, produto ou processo no mercado; apresentar breve descrição da participação de entidades externas no projeto.

\section{MOTIVAÇÃO PARA O EMPREENDEDORISMO OU SETOR PRODUTIVO/MERCADO DE TRABALHO:} \label{sec:motivação_empreendedorismo}
% Quadro do item 6.6 do Edital.
% Apresenta a possibilidade de geração de spin-off (empresa criada com o objetivo de explorar novos produtos ou serviços de base tecnológica ou inovadora que nasce a partir de ideias ou processos gerados numa organização já existente - por exemplo, uma Instituição de Ciência e Tecnologia?

\section{CRONOGRAMA:} \label{sec:cronograma}
% Compatível com a execução?
% Distribui adequadamente as tarefas em relação ao tempo previsto?

\begin{table}[ht]
    \centering
    \renewcommand{\arraystretch}{1.2} % Define a altura padrão das linhas
    \begin{tabularx}{\textwidth}{
        |>{\arraybackslash\RaggedRight}p{5cm}
        |>{\centering\arraybackslash}X
        |>{\centering\arraybackslash}X
        |>{\centering\arraybackslash}X
        |>{\centering\arraybackslash}X
        |>{\centering\arraybackslash}X
        |>{\centering\arraybackslash}X
        |>{\centering\arraybackslash}X
        |>{\centering\arraybackslash}X
        |>{\centering\arraybackslash}X
        |>{\centering\arraybackslash}X
        |>{\centering\arraybackslash}X
        |>{\centering\arraybackslash}X|
    }
        \hline
        \rowcolor{gray!10} &   \multicolumn{12}{c|}{\textbf{Mês/Ano de atividade}}  \\ \cline{2-13}
        \rowcolor{gray!10}\textbf{Atividade} & \textbf{1} & \textbf{2} & \textbf{3} & \textbf{4} & \textbf{5} & \textbf{6} & \textbf{7} & \textbf{8} & \textbf{9} & \textbf{10} & \textbf{11} & \textbf{12} \\ \hline
        Atividade A                &  &  &  &  &  &  &  &  &  &  &  &  \\ \hline
    \end{tabularx}
    \caption{Cronograma do projeto}
    \label{tab:cronograma}
\end{table}

\section{EXEQUIBILIDADE DO PROJETO:}
% Expressa a viabilidade do projeto por meio da relação: atividades a serem realizadas, tempo disponível para a execução e recursos necessários para sua realização?
% Indica os equipamentos e materiais necessários ao desenvolvimento do projeto?
% Em caso de necessidade de aquisição de equipamentos ou materiais, indica como serão captados os recursos?

\section{Justificativa para continuidade do projeto: (somente para os casos de projetos aprovados em editais anteriores)} \label{sec:justificativa}

\def\refname{REFERÊNCIAS}
\bibliography{referencias}
\end{document}
